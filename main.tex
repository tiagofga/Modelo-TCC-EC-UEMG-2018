% Modelo de Trabalho Acadêmico do curso de Engenharia da Computação em conformidade com
% ABNT NBR 14724:2011: Informação e documentação - Trabalhos acadêmicos 
% Autor Tiago Alves de Oliveira
% ------------------------------------------------------------------------
% ------------------------------------------------------------------------

\documentclass[
	12pt,				% tamanho da fonte
	openright,			% capítulos começam em página ímpar (insere página vazia caso preciso)
	oneside,			% para impressão em verso e anverso. Oposto a oneside
	a4paper,			% tamanho do papel. 
	brazil				% o último idioma é o principal do documento
	]{abntex2}

% ---
% Pacotes básicos 
% ---
\usepackage{lmodern}			% Usa a fonte Latin Modern			
\usepackage[T1]{fontenc}		% Selecao de codigos de fonte.
\usepackage[utf8]{inputenc}		% Codificacao do documento (conversão automática dos acentos)
\usepackage{lastpage}			% Usado pela Ficha catalográfica
\usepackage{indentfirst}		% Indenta o primeiro parágrafo de cada seção.
\usepackage{color}				% Controle das cores
\usepackage{graphicx}			% Inclusão de gráficos
\usepackage{microtype} 			% para melhorias de justificação
% ---
		
% ---
% Pacotes adicionais, usados apenas no âmbito do Modelo Canônico do abnteX2
% ---
\usepackage{lipsum}				% para geração de dummy text
% ---

% ---
% Pacotes de citações
% ---
\usepackage[brazilian,hyperpageref]{backref}	 % Paginas com as citações na bibl
\usepackage[alf]{abntex2cite}	% Citações padrão ABNT
\usepackage{monografia}

% Informações de dados para CAPA e FOLHA DE ROSTO
\titulo{Título do Seu Trabalho}
\autor{Tiago Alves de Oliveira}
\local{Divinópolis - Brasil}
\data{\today}
\orientador{Professor Orientador}
\coorientador{Professor Coorientador}
\instituicao{%
  Universidade Estadual de Minas Gerais -- UEMG
  \par
  Unidade Divinópolis
  \par
  Curso de Engenharia da Computação}
\tipotrabalho{Monografia}
% O preambulo deve conter o tipo do trabalho, o objetivo, 
% o nome da instituição e a área de concentração 
\preambulo{Monografia apresentada ao Curso de Engenharia de Computação da UEMG Unidade Divinópolis, como requisito parcial para obtenção do título de Bacharel em Engenharia da Computação, sob a orientação do Prof. XXXX}
% ---


% ---
% Configurações de aparência do PDF final

% alterando o aspecto da cor azul
\definecolor{blue}{RGB}{41,5,195}

% informações do PDF
\makeatletter
\hypersetup{
     	%pagebackref=true,
		pdftitle={\@title}, 
		pdfauthor={\@author},
    	pdfsubject={\imprimirpreambulo},
	    pdfcreator={LaTeX with abnTeX2},
		pdfkeywords={abnt}{latex}{abntex}{abntex2}{trabalho acadêmico}, 
		colorlinks=true,       		% false: boxed links; true: colored links
    	linkcolor=blue,          	% color of internal links
    	citecolor=blue,        		% color of links to bibliography
    	filecolor=magenta,      		% color of file links
		urlcolor=blue,
		bookmarksdepth=4
}
\makeatother
% --- 

% --- 
% Espaçamentos entre linhas e parágrafos 
% --- 

% O tamanho do parágrafo é dado por:
\setlength{\parindent}{1.3cm}

% Controle do espaçamento entre um parágrafo e outro:
\setlength{\parskip}{0.2cm}  % tente também \onelineskip

\makeindex

\begin{document}

% Retira espaço extra obsoleto entre as frases.
\frenchspacing 

%Elementos Pré Textuais
\imprimircapa
\imprimirfolhaderosto*

\include{fichaCatalografica}
%\include{errata} %Caso seu texto tenha errata descomentar esta linha
\include{folhaAprovacao}
\include{dedicatoria}
\begin{agradecimentos}
Os agradecimentos principais são direcionados 
\end{agradecimentos}
\include{epigrafe}


\pagenumbering{roman} % MODIFICANDO A FORMA DE NUMERAÇÃO DAS PÁGINAS
% inserir o sumario
\pdfbookmark[0]{\contentsname}{toc}
\tableofcontents*
\cleardoublepage
% inserir lista de ilustrações
\pdfbookmark[0]{\listfigurename}{lof}
\listoffigures*
\cleardoublepage
% inserir lista de tabelas
\pdfbookmark[0]{\listtablename}{lot}
\listoftables*
\cleardoublepage
% inserir lista de abreviaturas e siglas
\begin{siglas}
  \item[ABNT] Associação Brasileira de Normas Técnicas
  \item[abnTeX] ABsurdas Normas para TeX
\end{siglas}
% inserir lista de símbolos
\begin{simbolos}
  \item[$ \Gamma $] Letra grega Gama
  \item[$ \Lambda $] Lambda
  \item[$ \zeta $] Letra grega minúscula zeta
  \item[$ \in $] Pertence
\end{simbolos}

% RESUMOS
% resumo em português
\setlength{\absparsep}{18pt} % ajusta o espaçamento dos parágrafos do resumo
\begin{resumo}
 Seu resumo aqui

 \textbf{Palavras-chaves}: latex. abntex. editoração de texto.
\end{resumo}
% resumo em inglês
\begin{resumo}[Abstract]
 \begin{otherlanguage*}{english}
   
   O abstract vem aqui

   \vspace{\onelineskip}
 
   \noindent 
   \textbf{Key-words}: latex. abntex. text editoration.
 \end{otherlanguage*}
\end{resumo}

% ELEMENTOS TEXTUAIS
\textual
%Incluindo os capítulos
\pagenumbering{arabic} % MODIFICANDO A FORMA DE NUMERAÇÃO DAS PÁGINAS

% Introdução 
\chapter[Introdução]{Introdução}\label{capitulo1}
\addcontentsline{toc}{chapter}{Introdução}

Falar de 

\section{Considerações Iniciais}

\section{Objetivos do Trabalho}

\section{Estrutura da Monografia}

A presente monografia está dividida em cinco capítulos. No capítulo \ref{capitulo2}  são apresentados os trabalhos realizados. No capítulo \ref{capitulo3} são apresentadas todas as técnicas necessárias para que possa utilizar esse artifício junto aos objetivos do trabalho. No capítulo \ref{capitulo4} são apresentados a metodologia e desenvolvimento do trabalho. NO capítulo \ref{capitulo5} é apresentado o desenvolvimento do trabalho .No capítulo \ref{capitulo6} são apresentados os resultados e as análises dos mesmos. Por fim, no capítulo \ref{capitulo7} são apresentadas as conclusões e possíveis trabalhos futuros.
\chapter[Estado da Arte]{Estado da Arte}\label{capitulo2}
\addcontentsline{toc}{chapter}{Estado da Arte}
\chapter[Fundamentação Teórica]{Fundamentação Teórica}\label{capitulo3}
\addcontentsline{toc}{chapter}{Fundamentação Teórica}
\chapter[Metodologia ou Materiais e Métodos]{Metodologia ou Materiais e Métodos}\label{capitulo4}
\addcontentsline{toc}{chapter}{Metodologia ou Materiais e Métodos}
\chapter[Desenvolvimento]{Desenvolvimento}\label{capitulo5}
\addcontentsline{toc}{chapter}{Desenvolvimento}
\chapter[Análise dos Resultados]{Análise dos Resultados}\label{capitulo6}
\addcontentsline{toc}{chapter}{Análise dos Resultados}
\chapter[Trabalhos Futuros]{Trabalhos Futuros}\label{capitulo7}
\addcontentsline{toc}{chapter}{Trabalhos Futuros}

\section{Trabalhos Futuros}

% ----------------------------------------------------------
% ELEMENTOS PÓS-TEXTUAIS
% ----------------------------------------------------------
\postextual
% ----------------------------------------------------------

% ----------------------------------------------------------
% Referências bibliográficas
% ----------------------------------------------------------
\bibliography{bibliografia}

% ----------------------------------------------------------
% Glossário
% ----------------------------------------------------------
% Consulte o manual da classe abntex2 para orientações sobre o glossário.
%\glossary

% Apêndices
\begin{apendicesenv}

% Imprime uma página indicando o início dos apêndices
\partapendices

\chapter{1}
Texto

\chapter{2}
Texto

\end{apendicesenv}
% ---
% Anexos
\begin{anexosenv}

% Imprime uma página indicando o início dos anexos
\partanexos

\chapter{1}
Texto

\chapter{2}
Texto

\chapter{3}
Texto

\end{anexosenv}

%---------------------------------------------------------------------
% INDICE REMISSIVO
%---------------------------------------------------------------------
\phantompart
\printindex
%---------------------------------------------------------------------

\end{document}
